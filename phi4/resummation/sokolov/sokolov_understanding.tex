\documentclass[preprint,preprintnumbers,amsmath,amssymb]{revtex4}
\usepackage{amsmath,amssymb,amsfonts}
\usepackage{color,array}
\usepackage{dsfont}
\usepackage{slashed}
\usepackage{graphicx,graphics,epsfig}
\usepackage{bbm,bm}
\usepackage{psfrag}
\usepackage{hyperref}
\usepackage{ulem}
\usepackage{url}
\usepackage[utf8]{inputenc}
\usepackage[english,russian]{babel}

\baselineskip=0.9\baselineskip

\newcommand{\comment}[1]{} % многострочные комментарии

\usepackage{xcolor}
\definecolor{greenb}{rgb}{0.2,0.7,0.2}
%% Команды для покраски клеток таблицы
\newcommand{\red}[1]{\colorbox{red}{#1}}    % красим красным, например: \red{1.234}
\newcommand{\green}[1]{\colorbox{green}{#1}}% красим зелёным, например: \green{1.234}
% красим тёмно-зелёным с красной рамкой, например: \darkgreen{1.234}
\newcommand{\darkgreen}[1]{\fcolorbox{red}{greenb}{#1}}

\begin{document}

\title{Проверка результатов статьи \\ 
\href{http://arxiv.org/abs/1312.1062}{<<Critical Exponents in Two Dimensions and Pseudo-$\epsilon$ Expansion>>}}
\author{M. A. Nikitina}
\author{A. I. Sokolov}
\email{ais2002@mail.ru}
\affiliation{Department of Quantum Mechanics,
Saint Petersburg State University,
Ulyanovskaya 1, Petergof,
Saint Petersburg, 198504
Russia}
\date{\today}

\comment{
\begin{abstract}
The critical behavior of two-dimensional $n$-vector $\lambda\phi^4$ field model
is studied within the framework of pseudo-$\epsilon$ expansion approach.
Pseudo-$\epsilon$ expansions for Wilson fixed point location $g^*$ and critical
exponents originating from five-loop 2D renormalization group series are derived.
Numerical estimates obtained within Pad\'e and Pad\'e-Borel resummation procedures
as well as by direct summation are presented for $n = 1$, $n = 0$ and $n = -1$,
i. e. for physically interesting models which are exactly solvable.
The pseudo-$\epsilon$ expansions for $g^*$, critical exponents $\gamma$ and $\nu$
have small lower-order coefficients and slow increasing higher-order ones. As a
result, direct summation of these series with optimal cut off provides numerical
estimates that are no worse than those given by the resummation approaches
mentioned. This enables one to consider the pseudo-$\epsilon$ expansion technique
itself as some specific resummation method.
\end{abstract}

\pacs{05.10.Cc, 05.70.Jk, 64.60.ae, 64.60.Fr}
}
\maketitle

\section{Introduction}

Pseudo-$\epsilon$ expansion is known to be rather effective when used to
estimate numerical values of universal quantities characterizing critical
behavior of three-dimensional systems \cite{GZJ1980, GZJ1998, FHY2000, HID2004}.
Moreover, even in two dimensions, where original renormalization group (RG)
series are shorter and more strongly divergent, pseudo-$\epsilon$ expansion
technique is able to give good or satisfactory results \cite{GZJ1980, COPS2004,
S2005}. To obtain numerical estimates from pseudo-$\epsilon$ expansions one
applies a resummation technique since corresponding series have growing
higher-order coefficients, i. e. look divergent. In contrast to RG expansions in
fixed and $4-\epsilon$ dimensions, pseudo-$\epsilon$ expansions do not need in
advanced resummation procedures based on Borel transformation. As a rule, use of
simple Pad\'e approximants turns out to be sufficient to obtain proper numerical
estimates \cite{FHY2000, COPS2004, S2005}.

<$\dots$>
\comment{
In this paper, we study the critical behavior of two-dimensional $O(n)$-symmetric
systems within the frame of pseudo-$\epsilon$ expansion technique. The series for
the Wilson fixed point location $g*$ and critical exponents originating from the
five-loop RG expansions will be derived for arbitrary order parameter
dimensionality $n$. The pseudo-$\epsilon$ expansions obtained will be analysed in
detail for $n = 1$, $n = 0$ and $n = -1$, i. e. for the cases corresponding to
physically realizable systems with exactly known critical exponents \cite{N1982,
N1984, FQS1984}. These systems may be considered as testbeds for clarification of
the numerical effectiveness of various approximation schemes including RG
perturbation theory and the method of pseudo-$\epsilon$ expansion.} Numerical
estimates for critical exponents will be extracted from the pseudo-$\epsilon$
expansions by means of Pad\'e and Pad\'e-Borel resummation techniques as well as
by direct summation. The latter approach will be applied under the assumption
that the best numerical results may be obtained by means of cutting divergent
pseudo-$\epsilon$ expansions off by smallest terms, i. e. applying the procedure
valid for asymptotic series.

\section{Pseudo-$\epsilon$ expansions for general $n$}

<$\dots$>
\comment{
The critical behavior of two-dimensional systems with $O(n)$-symmetric vector
order parameters is described by Euclidean field theory with the Hamiltonian:
\begin{equation}
\label{model}
H =
\int d^2x \Biggl[{1 \over 2}( m_0^2 \varphi_{\alpha}^2
 + (\nabla \varphi_{\alpha})^2)
+ {\lambda \over 24} (\varphi_{\alpha}^2)^2 \Biggr] ,
\end{equation}
where $\varphi_{\alpha}$ is a real $n$-vector field, bare mass squared $m_0^2$
being proportional to $T - T_c^{(0)}$, $T_c^{(0)}$ -- phase transition temperature
in the absence of order parameter fluctuations. The $\beta$-function and the
critical exponents for the model (1) have been calculated within the massive
theory \cite{OS2000, COPS2004}, with the Green function, the four-point vertex
and the $\phi^2$ insertion normalized in a conventional way:

\begin{eqnarray}
\label{norm}
G_R^{-1} (0, m, g_4) = m^2 , \qquad \quad
{{\partial G_R^{-1} (p, m, g_4)} \over {\partial p^2}}
\bigg\arrowvert_{p^2 = 0} = 1 , \\
\nonumber
\Gamma_R (0, 0, 0, m, g) = m^2 g_4, \qquad \quad
\Gamma_R^{1,2} (0, 0, m, g_4) = 1.
\end{eqnarray}
}
Starting from the five-loop RG expansion for $\beta$-function \cite{OS2000},
we replace the linear term in this expansion with $\tau g$, calculate the
Wilson fixed point coordinate $g^*$ as series in $\tau$, and arrive to the
following expression:
\begin{eqnarray}
\label{g-tau}
g* &=& \tau + {\tau^2 \over (n + 8)^2} \biggl( 10.33501055~n + 47.67505273 \biggr)
\nonumber \\
&+& {\tau^3 \over (n + 8)^4} \biggl(- 5.00027593~n^3 + 24.4708201~n^2 + 253.297221~n
+ 350.808487 \biggr) \ \
\nonumber \\
&+& {\tau^4 \over (n + 8)^6} \biggl( 0.088842906~n^5 - 77.270445~n^4
+ 45.052398~n^3 + 3408.2839~n^2
\nonumber \\
&+& 14721.151~n + 27649.346 \biggr) - {\tau^5 \over (n + 8)^8} \biggl(- 0.00407946~n^7
- 0.305739~n^6
\nonumber \\
&+& 1464.58~n^5 + 11521.4~n^4 + 98803.3~n^3 + 794945~n^2 + 3146620~n + 4734120 \biggr). \ \
\end{eqnarray}
Substituting this expansion into five-loop RG series for critical exponents $\gamma$
and $\eta$ \cite{OS2000, COPS2004} we obtain:

\begin{eqnarray}
\label{gamma-tau}
\gamma^{-1} &=& 1 - {\tau~ (n + 2) \over (n + 8)}
- {\tau^2 \over (n + 8)^3} \biggl( 6.95938160~n^2 + 34.58878428~n + 41.34004218 \biggr)
\nonumber \\
&+& {\tau^3 \over (n + 8)^5} \biggl(0.338391156~n^4 - 53.7045862~n^3 - 181.874852~n^2
+ 471.838217~n
\nonumber \\
&+& 1236.12490 \biggr) - {\tau^4 \over (n + 8)^7} \biggl(- 0.23015013~n^6 + 21.848143~n^5
+ 1537.3578~n^4
\nonumber \\
&+& 12405.258~n^3 + 41577.259~n^2 + 75410.316~n + 59869.804 \biggr) \ \
\nonumber \\
&+& {\tau^5 \over (n + 8)^9} \biggl(0.115623~n^8 + 17.8566~n^7 + 83.1552~n^6 + 14850.5~n^5
- 84964.7~n^4
\nonumber \\
&+& 318099~n^3 + 3766200~n^2 + 10883700~n + 10128500 \biggr). \ \
\end{eqnarray}

\begin{eqnarray}
\label{eta-tau}
\eta &=& {\tau^2 \over (n + 8)^2}~0.9170859698 \biggl(n + 2 \biggr)
\nonumber \\
&+& {\tau^3 \over (n + 8)^4} \biggl(- 0.0546089776~n^3 + 17.9732248~n^2 + 120.114155~n
+ 167.898539 \biggr) \ \
\nonumber \\
&+& {\tau^4 \over (n + 8)^6} \biggl(- 0.092684458~n^5 - 8.2910597~n^4 - 174.43187~n^3
+ 2120.0408~n^2
\nonumber \\
&+& 7034.6638~n + 7114.3103 \biggr) + {\tau^5 \over (n + 8)^8} \biggl(- 0.0709196~n^7
- 5.60392~n^6 - 250.874~n^5
\nonumber \\
&+& 1312.68~n^4 + 36126.0~n^3 + 201476~n^2 + 470848~n + 396119 \biggr). \ \
\end{eqnarray}


<$\dots$>
\comment{
Pseudo-$\epsilon$ expansions for other critical exponents can be deduced from (4), (5)
using well-known scaling relations. The series for the correlation length exponent
$\nu$, for example, results from the formula
\begin{equation}
\label{scaling}
\gamma = \nu (2 - \eta).
\end{equation}
}

\section{Critical exponents for $n=1$, $n=0$ and $n =-1$}

<$\dots$>
\comment{
It is of major interest to analyze numerical results given by the
obtained expansions for the values of $n$ under which the model
(1) describes physically realizable systems and is exactly
solvable. That is why further we concentrate on the cases $n=1$,
$n=0$, and $n =-1$. Pseudo-$\epsilon$ expansions for critical
exponents we'll deal with are as follows:
}

\begin{center}
\textbf{$n = 1$}
\end{center}
\begin{eqnarray}
\gamma = 1 + \tau/3 + 0.224812357 \tau^{2}
+ 0.087897190 \tau^{3}+0.086443008\tau^{4}-0.0180209 \tau^{5}.
\end{eqnarray}
\begin{eqnarray}
\gamma^{-1} = 1 - \tau/3 - 0.113701246 \tau^2
+ 0.024940678 \tau^3-0.039896059 \tau^4+0.0645210 \tau^5.
\end{eqnarray}
\begin{eqnarray}
\nu = 1/2 + \tau/6 + 0.120897626 \tau^{2}
+ 0.0584361287 \tau^{3} + 0.056891652 \tau^{4} + 0.00379868 \tau^{5}.
\end{eqnarray}
\begin{eqnarray}
\nu^{-1} = 2 - 2\tau/3 - 0.261368281 \tau^{2}
+ 0.0145750797 \tau^{3} - 0.091312521 \tau^{4} + 0.118121 \tau^{5}.
\end{eqnarray}
\begin{eqnarray}
\eta = 0.0339661470 {\tau}^{2}+0.0466287623 {\tau}^{3}
+ 0.030925471 {\tau}^{4}+0.0256843 {\tau}^{5}.
\end{eqnarray}

\begin{center}
\textbf{$n = 0$}
\end{center}
\begin{eqnarray}
\gamma = 1 + \tau/4 + 0.143242270 \tau^{2} + 0.018272597 \tau^{3}
+ 0.035251118 \tau^{4}-0.0634415\tau^{5}.
\end{eqnarray}
\begin{eqnarray}
\gamma^{-1} = 1- \tau/4 - 0.080742270\tau^{2}+0.037723538 \tau^{3}
- 0.028548147 \tau^{4}+0.0754631 \tau^{5}.
\end{eqnarray}
\begin{eqnarray}
\nu = 1/2 + \tau/8 + 0.0787857831 \tau^{2}
+ 0.0211750671 \tau^{3} + 0.028101050 \tau^{4} - 0.0222040 \tau^{5}.
\end{eqnarray}
\begin{eqnarray}
\nu^{-1} = 2 - \tau/2 - 0.190143132 \tau^{2}
+ 0.0416212976 \tau^{3} - 0.071673308 \tau^{4} + 0.136330 \tau^{5}.
\end{eqnarray}
\begin{eqnarray}
\eta = 0.0286589366 {\tau}^{2} + 0.0409908542 {\tau}^{3}
+ 0.027138940 {\tau}^{4} + 0.0236106 {\tau}^{5}.
\end{eqnarray}

\begin{center}
\textbf{$n = -1$}
\end{center}
\begin{eqnarray}
\gamma = 1 + \tau/7 + 0.060380873 \tau^{2}-0.023532210 \tau^{3}
+ 0.012034268 \tau^{4}-0.0638772 \tau^{5}.
\end{eqnarray}
\begin{eqnarray}
\gamma^{-1} = 1- \tau/7 - 0.039972710 \tau^2+0.037868436 \tau^3
- 0.018392201 \tau^{4}+ 0.0649966 \tau^{5}.
\end{eqnarray}
\begin{eqnarray}
\nu = 1/2 + \tau/14 + 0.0348693698 \tau^{2}
- 0.00424514372 \tau^{3} + 0.011608435 \tau^{4} - 0.0268913 \tau^{5}.
\end{eqnarray}
\begin{eqnarray}
\nu^{-1} = 2 - 2\tau/7 - 0.0986611527 \tau^{2}
+ 0.0510003794 \tau^{3} - 0.049264800 \tau^{4} + 0.116842 \tau^{5}.
\end{eqnarray}
\begin{eqnarray}
\eta = 0.0187160402 {\tau}^{2} + 0.0274103364 {\tau}^{3}
+ 0.017144702 {\tau}^{4} + 0.0159901 {\tau}^{5}.
\end{eqnarray}

The expansions for "big" critical exponents $\gamma$, $\nu$ and for their
inverses are seen to possess coefficients which begin to grow from certain
terms indicating that these series are divergent. Moreover, they are not
alternative, i. e. their coefficients have irregular signs. At the same time,
lower-order coefficients in expansions (7)--(10), (12)--(15) and (17)--(20)
decrease, and decrease more rapidly than their counterparts in the original
RG series. This enables one to consider them as suitable for some resummation
and getting proper numerical estimates.

The structure of pseudo-$\epsilon$ expansions for "small" exponent $\eta$ is
quite different. These series have positive coefficients of the same order of
magnitude what makes questionable an applicability of any procedure employed
nowadays for resummation of diverging RG series.

To demonstrate a power of various resummation techniques and to clear up to
what extent they are necessary in the case considered we present below numerical
results given by several relevant procedures. Namely, we evaluate critical
exponents $\gamma$ and $\nu$ for $n=1$, $n=0$ and $n =-1$ by means of the Pad\'e
resummation, by Pad\'e-Borel resummation of the pseudo-$\epsilon$ expansions for
exponents themselves and for their inverses, and by direct summation of the series
(7)--(10), (12)--(15) and (17)--(20). Direct summation is performed under the
assumption that one can get the best numerical estimates cutting off divergent
pseudo-$\epsilon$ expansions by smallest terms, i. e. adopting the procedure
true for asymptotic series.

The results thus obtained are collected in Table I. Along with
pseudo-$\epsilon$ expansion estimates the exact values of critical
exponents and the estimates originating from five-loop RG series
\cite{OS2000} are presented here for comparison. Numerical values
of the Fisher exponent given by direct summation of series (11),
(16), and (21) are also included to give an idea about the level
of accuracy of the pseudo-$\epsilon$ expansion technique in the
case of small critical exponent.

Before discussing content of Table 1 we present some details concerning the critical
exponent values obtained. In principle, Pad\'e resummation procedure is known to be
rather effective when applied to pseudo-$\epsilon$ expansions for critical exponents
and other universal quantities \cite{FHY2000, HID2004, COPS2004, S2005}.
It demonstrates, as a rule, good convergence if one deals with high enough orders in
$\tau$. In two dimensions, however, the numbers given by Pad\'e resummed expansions
may converge to the values differing considerably from their exact counterparts.
Pad\'e triangles presented below illustrate this situation. The first one (Table II)
shows most favorable situation - exponent $\nu$ at $n = 0$ - when numerical
estimates regularly converge to the true value $\nu = 0.75$. The second example
(Table III) demonstrates that good convergence may not result in quite good numerical
estimate: the asymptotic value $\nu = 0.606$ differs appreciably from the exact one
$\nu = 0.625$ for $n = -1$. At last, from Table IV (the exponent $\gamma$, $n = 0$)
one can see that fair convergence does not guarantee satisfactory numerical results
- the estimates in this Table concentrate near 1.435, i. e. far from the exact value
1.34375.

Similar situation takes place when we address Pad\'e-Borel
resummation technique. This procedure may result in either good
numerical results or unsatisfactory ones depending on the critical
exponent evaluated and on the value of $n$. Tables V-VII
illustrate this statement. Pad\'e-Borel resummation of the
pseudo-$\epsilon$ expansion of the inverse exponent $\nu$ for $n =
0$ gives quite good numerical estimates (Table V) while estimates
of $\nu$ for $n = -1$ and $\gamma$ for $n = 0$ via inverse
expansions (Tables VI and VII) "miss" the exact values. Moreover,
Pad\'e-Borel triangles for exponents $\gamma$ and $\nu$ themselves
at $n = 1$ and some others turn out to be half-empty because many
Pad\'e approximants are spoilt by "dangerous" (positive axis)
poles.

\section{To resum or not to resum?}

Let us return back to Table I. As is seen, numerical estimates provided by Pad\'e
and Pad\'e-Borel resummation techniques may i) be considerably scattered and
ii) differ from the exact values no less than numbers given by direct summation
of pseudo-$\epsilon$ expansions and of corresponding inverse series. On the other
hand, direct summation of these expansions generates an iteration procedure which
rapidly converge to asymptotic values that are as close to the exact ones as those
obtained within various resummation methods. Figures 1-4, where partial sums of
series (7-10) and (12-15) are depicted as functions of the order in $\tau$,
illustrate the situation. Filled rounds and triangles mark the points of optimal
cut off, i. e. the order from which the coefficients of pseudo-$\epsilon$ expansions
start to grow. Figures 1, 2 show the favorable cases when approximate values almost
coincide with exact ones. Figures 3, 4, to the contrary, show most unfavorable
regimes when the difference between approximate and exact values turns out to reach
0.1. Analogous level of accuracy is observed when small critical exponent $\eta$ is
estimated. Indeed, the direct summation of the pseudo-$\epsilon$ expansion (see
Table I) and application of the resummation techniques result in numbers grouping
around the exact values within the range of order of 0.1.

So, the resummation of pseudo-$\epsilon$ expansions for two dimensional models
practically does not improve numerical estimates of critical exponents. Moreover,
the direct summation leads to approximate values which are as accurate as those
resulting from original five-loop RG series (see Table I). This enables us to
conclude that estimating critical exponents in two dimensions within the
pseudo-$\epsilon$ expansion approach one can use the simplest possible way to
process the series - direct summation with optimal cut off.

In this sense the pseudo-$\epsilon$ expansion itself may be considered as some
special resummation method. There are two reasons for such a point of view. First,
this approach transforms strongly divergent field-theoretical RG expansions into
power series with much smaller lower-order coefficients and much slower increasing
higher-order ones. Second, the physical value of expansions parameter $\tau$ is
equal to 1, while the Wilson fixed point coordinate $g^*$ playing analogous role
within field-theoretical RG approach is almost two times bigger in two dimensions
($g^* = 1.84 - 1.86$ for $n = 1, 0, -1$ \cite{OS2000}). This difference looks
essential, especially keeping in mind importance of higher-order terms.

<$\dots$>
\comment{
\section{Conclusion}

To summarize, we have calculated pseudo-$\epsilon$ expansions for dimensionless
effective coupling constant $g^*$ and critical exponents of 2D Euclidean $n$-vector
field theory up to $\tau^5$ order. Numerical estimates of critical exponents for
models with $n = 1, 0, -1$ exactly solvable at criticality have been found using
Pad\'e and Pad\'e-Borel resummation techniques as well as by direct summation with
optimal cut off. Comparison of the results obtained with each others and with their
exact counterparts has shown that direct summation of pseudo-$\epsilon$ expansions
provides, in general, numerical estimates that are no worse than those given by
resummation approaches mentioned. This implies that the pseudo-$\epsilon$ expansion
approach may be thought of as some specific resummation technique.
}

\begin{thebibliography}{999}

\bibitem{GZJ1980} J. C. Le Guillou and J. Zinn-Justin, Phys. Rev. B \textbf{21}, 3976, (1980).

\bibitem{GZJ1998} R. Guida and J. Zinn-Justin, J. Phys. A \textbf{31}, 8103, (1998).

\bibitem{FHY2000} R. Folk, Yu. Holovatch, and T. Yavorskii, Phys. Rev. B \textbf{62}, 12195, (2000).

\bibitem{HID2004} Yu. Holovatch, D. Ivaneiko, and B. Delamotte, J. Phys. A \textbf{37}, 3569, (2004).

\bibitem{COPS2004} P. Calabrese, E. V. Orlov, D. V. Pakhnin, and A. I. Sokolov, Phys. Rev. B
\textbf{70}, 094425, (2004).

\bibitem{S2005} A. I. Sokolov, Fiz. Tverd. Tela \textbf{47}, 2056, (2005) [Phys. Sol. State
\textbf{47}, 2144, (2005)]

\bibitem{N1982} B. Nienhuis. ~Phys. Rev. Lett. \textbf{49}, ~1062 ~(1982).

\bibitem{N1984} B. Nienhuis. ~J. Stat. Phys. \textbf{34}, ~731 ~(1984).

\bibitem{FQS1984} D. Friedan, ~Z. Qiu, ~S. Shenker. ~Phys. Rev. Lett. \textbf{52}, ~1575 ~(1984).

\bibitem{OS2000} E. V. Orlov and A. I. Sokolov, Fiz. Tverd. Tela \textbf{42}, 2087, (2000)
[Phys. Sol. State \textbf{42}, 2151, (2000)]

\end{thebibliography}

\begin{table}[t]
\caption{Numerical values of critical exponents for $n=1$, $n=0$ and $n=-1$ found
by direct summation (DS) of the pseudo-$\epsilon$ expansions (see the text) and
of corresponding inverse series (DS$^{-1}$), by Pad\'e resummation of the series
for $\gamma$ and $\nu$, and by Pad\'e-Borel resummation of the pseudo-$\epsilon$
expansions and of their inverses using Pad\'e approximants [2/3] and [3/2]. Pad\'e
estimates presented are averaged over those given by [2/3] and [3/2] approximants.
Exact values of critical exponents and the estimates obtained from original
five-loop renormalization-group series \cite{OS2000} are also presented for
comparison.}
\label{tab1}

\begin{tabular}{|*{9}{c|}}\hline
\multicolumn{9}{|c|}{Critical exponents (CE) for various $n$.} \\ \hline
~CE~ & exact & ~DS~ & ~DS$^{-1}~ $ & ~Pad\'e~ & PB$_{[2/3]}$ &
(PB$^{-1}$)$_{[2/3]}$ & (PB$^{-1}$)$_{[3/2]}$ &~5-loop RG~\\ \hline
\multicolumn{9}{|c|}{$n=1$} \\ \hline
$\gamma$ & 1.75 & \green{~1.7145~} & \darkgreen{~1.7304~$^{N=3}$} & \green{~1.775~} & \green{~1.6105~} & \green{~1.7746~} &
$\frac{\red{~~~--~~~}}{1.7815}$ & ~1.790~ \\ \hline
$\nu$    & 1 & \green{~0.9067~} & \darkgreen{~0.9204~$^{N=3}$} & \green{~0.959~} & \green{~0.8136~} & \green{~0.9652~} & \red{--} &
~0.966~ \\ \hline
$\eta$   & 0.25 & \green{~0.1372~} & \green{~}  & $\frac{\red{None}}{0.044}$ &  \red{~} &  \green{~} & \green{~}  & ~0.146~ \\ \hline
%%%%%%%%%%
%% n = 0 %
%%%%%%%%%%
\multicolumn{9}{|c|}{$n=0$} \\ \hline
$\gamma$ & ~1.34375~$\left(\frac{43}{32}\right)$ & \darkgreen{~1.4115~$^{N=3}$} & \darkgreen{~1.4740~$^{N=4}$} & \green{~1.435~} & \green{~1.3804~} &
\green{~1.4285~} & \green{~1.4429~} & ~1.449~ \\ \hline
$\nu$    & 0.75 &  \darkgreen{~0.7250~$^{N=3}$} &  \darkgreen{~0.7399~$^{N=3}$} & \green{~0.753~} & \green{~0.7069~} & \green{~0.7514~} &
$\frac{\red{~~~--~~~}}{0.755}$ & ~0.774~ \\ \hline
$\eta$   & 0.20833~$\left(\frac{5}{24}\right)$ & \green{~0.1204~} & \green{~}  & $\frac{\red{None}}{0.069}$ & \red{~}  &  \green{~} &  \green{~} & ~0.128~ \\ \hline
%%%%%%%%%%
%% n =-1 %
%%%%%%%%%%
\multicolumn{9}{|c|}{$n=-1$} \\ \hline
$\gamma$ & ~1.15625~$\left(\frac{37}{32}\right)$& \darkgreen{~1.1917~$^{N=4}$} & \darkgreen{~1.1952~$^{N=4}$} & \green{~1.192~} & \green{~1.1641~} &
\green{~1.1843~}\footnote{клетки, соответствующие значениям [2/3] и [3/2], перепутаны местами} 
& $\frac{\red{~~~--~~~}}{1.202Warn}$\footnote{клетки, соответствующие значениям [2/3] и [3/2], перепутаны местами} & ~1.184~ \\ \hline
$\nu$    & 0.625 & \darkgreen{~0.6021~$^{N=3}$} & \darkgreen{~0.6183~$^{N=4}$} & \green{~0.606~} & \green{~0.5945~} & \green{~0.6054~} &
\green{~0.6076~} & ~0.617~ \\ \hline
$\eta$   & 0.15 & \green{~0.0793~} & \green{~}  & $\frac{\red{None}}{0.055}$ & \red{~}  &  \green{~} & \green{~}  & ~0.082~ \\ \hline
\multicolumn{9}{|c|}{Легенда:} \\ \hline
\green{~1.23~} & \multicolumn{8}{l|}{значение совпадает с результатом Соколова} \\ \hline
\darkgreen{~1.23~} & \multicolumn{8}{l|}{понятно, как Соколов это получил (он взял только $N$ членов разложения)} \\ \hline
\red{~1.23~} & \multicolumn{8}{l|}{значение НЕ совпадает с результатом Соколова} \\ \hline
\end{tabular}
\end{table}

\begin{table}[t]
\caption{Pad\'e table originating from pseudo-$\epsilon$ expansion (14) for
critical exponent $\nu$ at $n=0$. The exact value of this critical exponent
is equal to 0.75.}
\label{tab2}
\renewcommand{\tabcolsep}{0.4cm}
\begin{tabular}{|*{7}{c|}}              \hline
              L/M & 0 & 1 & 2 & 3 & 4 & 5 \\ \hline
              0 & \green{0.500} & \green{0.625} & \green{0.704} & \green{0.725} & \green{0.753} & \green{0.731} \\ \hline
              1 & \green{0.667} & \green{0.838} & \green{0.733} & \green{0.639} & \green{0.741} &       \\ \hline
              2 & \green{0.763} & \green{0.744} & \green{0.763} & \green{0.752} &       &       \\ \hline
              3 & \green{0.740} & \green{0.755} & \green{0.754} &       &       &       \\ \hline
              4 & \green{0.781} & \green{0.754} &      &       &       &       \\ \hline
              5 & \green{0.706} &       &      &       &       &
               \\ \hline
\end{tabular}
\end{table}

\begin{table}[t]
\caption{Pad\'e triangle for pseudo-$\epsilon$ expansion (19) of exponent
$\nu$ at $n = -1$. The exact $\nu$ value equals 0.625.}
\label{tab3}
\renewcommand{\tabcolsep}{0.4cm}
\begin{tabular}{|*{7}{c|}}
              \hline
              L/M & 0 & 1 & 2 & 3 & 4 & 5 \\ \hline
              0 & \green{0.500} & \green{0.571} & \green{0.606} & \green{0.602} & \green{0.614} & \green{0.587} \\ \hline
              1 & \green{0.583} & \green{0.640} & \green{0.603} & \green{0.605} & \green{0.606} &      \\ \hline
              2 & \green{0.619} & \green{0.606} & \green{0.610} & \green{0.606} &     &      \\ \hline
              3 & \green{0.600} & \green{0.609} & \green{0.607} &      &     &      \\ \hline
              4 & \green{0.618} & \green{0.605} &      &      &     &      \\ \hline
              5 & \green{0.577} &       &      &      &     & 
               \\ \hline
\end{tabular}
\end{table}

\begin{table}[t]
\caption{Pad\'e table for pseudo-$\epsilon$ expansion (12) of exponent $\gamma$
at $n = 0$. The exact exponent value is 1.34375.}
\label{tab4}
\renewcommand{\tabcolsep}{0.4cm}
\begin{tabular}{|*{7}{c|}}
              \hline
              L/M & 0 & 1 & 2 & 3 & 4 & 5 \\ \hline
              0 & \green{1.000} & \green{1.25 } & \green{1.393} & \green{1.412} & \green{1.447} & \green{1.383} \\ \hline
              1 & \green{1.333} & \green{1.585} & \green{1.414} & \green{1.374} & \green{1.424} &       \\ \hline
              2 & \green{1.494} & \green{1.439} & \green{1.449} & \green{1.429} &       &       \\ \hline
              3 & \green{1.414} & \green{1.448} & \green{1.441} &       &       &       \\ \hline
              4 & \green{1.474} & \green{1.430} &       &       &       &       \\ \hline
              5 & \green{1.326} &       &       &       &       &
               \\ \hline
\end{tabular}
\end{table}

\begin{table}[t]
\caption{Pad\'e-Borel table for pseudo-$\epsilon$ expansion of $\nu^{-1}$ at $n = 0$.
The exact exponent value equals 0.75. Some estimates are absent because corresponding
Pad\'e approximants turn out to be spoilt by "dangerous" (positive axis) poles.}
\label{tab5}
\renewcommand{\tabcolsep}{0.4cm}
\begin{tabular}{|*{7}{c|}}\hline
L/M& 0& 1 & 2 & 3 & 4 & 5\\ \hline
0 & \green{~~0.5~~}&\green{ 0.6058} & \green{0.6555} & \green{0.6762} & \green{0.6888} & \green{0.6954} \\ \hline
1 & \green{0.6667} & \red{~~~--~~~} & \green{0.7170} & \red{~~--~~} & \green{0.7145} & \\ \hline
2 & \green{0.7634} &\green{ 0.7449} & \red{~~--~~} & \green{0.7514}& & \\ \hline
3 & \green{0.7399} &\green{ 0.7538} & $\frac{\red{~~~--~~~}}{0.755}$ & & & \\ \hline
4 & \green{0.7814} &\green{ 0.7549} & & & & \\ \hline
5 & \green{0.7061} & & & & & \\ \hline
\end{tabular}
\end{table}

\begin{table}[t]
\caption{The same as Table V for $n = -1$. The exact value of $\nu$ is equal to 0.625.}
\label{tab6}
\renewcommand{\tabcolsep}{0.4cm}
\begin{tabular}{|*{7}{c|}}\hline
L/M & 0 & 1 & 2 & 3 & 4 & 5\\ \hline
0 & \green{~~0.5~} & \green{0.5640} & \green{0.5903} & \green{0.5961} & \green{0.6012} & $\frac{\red{~~~--~~~}}{0.6009}$ \\ \hline
1 & \green{0.5833} & \red{~~~--~~~} & \green{0.5990} & \red{~~~--~~~} & \green{0.6009} & \\ \hline
2 & \green{0.6190} & \green{0.6072} & $\frac{\red{~~~--~~~}}{0.6088}$ & \green{0.6054} & & \\ \hline
3 & \green{0.6000} & \green{0.6086} & \green{0.6076} & & & \\ \hline
4 & \green{0.6183} & \green{0.6059} & & & & \\ \hline
5 & \green{0.5766} & & & & & \\ \hline
\end{tabular}
\end{table}

\begin{table}[t]
\caption{Pad\'e-Borel triangle for pseudo-$\epsilon$ expansion of $\gamma^{-1}$ at $n = 0$.
The exact exponent value is 1.34375. Absent estimates are due to Pad\'e approximant
dangerous poles.}
\label{tab7}
\renewcommand{\tabcolsep}{0.4cm}
\begin{tabular}{|*{7}{c|}}\hline
L/M & 0 & 1 & 2 & 3 & 4 & 5 \\ \hline
0 & \green{~~~1~~~} & $\frac{\red{1.3907}}{1.2116}$  & \green{1.3055} & \green{1.3411} & \green{1.3622} & \green{1.3721} \\ \hline
1 & \green{1.3333} & \red{~~~--~~~} & \green{1.3946} & \red{~~~--~~~} & \green{1.3907} & \\ \hline
2 & \green{1.4942} & \green{1.4424} & $\frac{\red{~~~--~~~}}{1.4461}$ & \green{1.4285} & & \\ \hline
3 & \green{1.4145} & \green{1.4458} & \green{1.4429} & & & \\ \hline
4 & \green{1.4740} & \green{1.4320} & & & & \\ \hline
5 & \green{1.3264} & & & & & \\ \hline
\end{tabular}
\end{table}

<$\dots$>
\comment{
\begin{figure}
\begin{center}
%\includegraphics[width=\linewidth]{fig1.eps}
\caption{The values of critical exponent $\gamma$ for $n = 1$ as functions of the order
in $\tau$ obtained by direct summation of pseudo-$\epsilon$ expansions (7), (8). Filled
round and triangle mark the points of optimal cut off, i. e. the orders at which
coefficients of the series finish to decrease.}
\label{fig1}
\end{center}
\end{figure}

\begin{figure}
\begin{center}
%\includegraphics[width=\linewidth]{fig2.eps}
\caption{The same as in Fig. 1 for the exponent $\nu$ at $n = 0$ (series (14), (15)).}
\label{fig2}
\end{center}
\end{figure}

\begin{figure}
\begin{center}
%\includegraphics[width=\linewidth]{fig3.eps}
\caption{Critical exponent $\gamma$ at $n = 0$ as function of order in $\tau$ obtained
by direct summation of series (12), (13).}
\label{fig3}
\end{center}
\end{figure}

\begin{figure}
\begin{center}
%\includegraphics[width=\linewidth]{fig4.eps}
\caption{The same as in Fig. 3 for series (9), (10)).}
\label{fig4}
\end{center}
\end{figure}
}

\end{document}
