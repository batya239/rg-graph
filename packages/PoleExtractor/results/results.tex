\documentclass[10pt,a4paper]{article}
\usepackage[utf8x]{inputenc}
\usepackage{ucs}
\usepackage[russian]{babel}
\usepackage[OT1]{fontenc}
\usepackage[intlimits,sumlimits]{amsmath}
\usepackage{amsfonts}
\usepackage{amssymb}
\usepackage{color}
\usepackage{tikz}
\usetikzlibrary{intersections}
\usetikzlibrary{snakes}
%
\input{diagram_commands}
%
\title{Результаты вычислений программы Pole Extractor}
\author{Глеб Довженко.}
\begin{document}
\maketitle
%
\section{Постановка задачи и обозначения.}
%
\paragraph{Задача.} Сосчитать константы ренормировки $Z_{i}$ и РГ-функции $\beta_{i}$, $\gamma_i$ для теорий $\varphi^3$, $\varphi^4$ в размерной регуляризации $d = 6-2\epsilon$, $d=4-2\epsilon$ соответственно. 
%
\paragraph{Обозначения.}
\begin{itemize}
\item Расходящиеся части (на примере $\varphi^3$):\\У 2-хвосток:
\begin{gather}
\textrm{Пр. расх.} \left\{ \diagram{e11-e-}{straight} \right\} \equiv  \diagram{e11-e-}{dots} + p^{2} \diagram{e11-e-}{wavy},\\
\diagram{e11-e-}{dots} = \diagram{e11-e-}{straight} \Big|_{p=0,\ (\mu^2 / \tau) = 1}, \\
\diagram{e11-e-}{wavy} = \partial_{p^{2}} \diagram{e11-e-}{straight} \Big|_{p \rightarrow \infty,\ (\mu^2 / \tau) = 1}.
\end{gather}
У 3-хвосток:
\begin{gather}
\textrm{Пр. расх.} \left\{ \diagram{e12-e3-e-}{straight} \right\} \equiv \diagram{e12-e3-e-}{dots} \ , \\
\diagram{e12-e3-e-}{dots} = \diagram{e12-e3-e-}{straight} \Big|_{p=0,\ (\mu^2 / \tau) = 1}. 
\end{gather}
\item $Z$, $L$, $K$, $R^{\prime}$:\\Для теории $\varphi^3$:
\begin{gather}
Z_{1} = 1 + \left[ L_{2} \Gamma_{2} \right] = (Z_{\varphi})^{2}, \\
Z_{3} = 1 - \left[ L_{3} \Gamma_{3} \right] = Z_{g}\cdot(Z_{\varphi})^{3},\\
\left[ L_{2} \Gamma_{2} \right] = KR^{\prime} \ \diagram{undef2tails}{wavy} \ ,\\
 \left[ L_{3} \Gamma_{3} \right] = KR^{\prime} \ \diagram{undef3tails}{dots} \ .
\end{gather}
\end{itemize}
%
\section{Теория $\varphi^3$.}
%
\paragraph{1 петля.} Чтобы сосчитать 1-петлевое приближение искомых величин, нужны только сильно связные 1-петлевые 2- и 3-хвостки.
\begin{gather}
\diagram{e12-e3-e-}{dots} = \frac{||3 - \epsilon||\epsilon||}{4},\\
\diagram{e11-e-}{dots} = \frac{||3 - \epsilon||-1 + \epsilon||}{2},\\
\diagram{e11-e-}{wavy} = \frac{||\epsilon||}{2} \cdot \left(\frac{||4 - \epsilon||}{3\cdot(3 -\epsilon)} - \frac{||3 - \epsilon||}{2} \right) .
\end{gather}
%
\paragraph{2 петли.} Для 2-петлевого приближения нужны 1-петлевые 2-, 3-хвостки с точкой
\begin{gather}
\diagram{e12-e3-e3-e-}{dots} = \frac{||3 - \epsilon||1 + \epsilon||}{12\tau}, \\
\diagram{e12-e3-e-}{dots} = \frac{||3 - \epsilon||0 + \epsilon||}{4},\\
\diagram{e12-e2--}{wavy} = \frac{||1+\epsilon||}{12}\cdot \left( \frac{||4-\epsilon||}{6-2\epsilon} - ||3-\epsilon|| \right), 
\end{gather}
и сильно связные 2-петлевые 2- и 3-хвостки.
\paragraph{3 петли.}
\section{Теория $\varphi^4$.}
\paragraph{1 петля.}
\paragraph{2 петли.}
\paragraph{3 петли.}
\end{document}